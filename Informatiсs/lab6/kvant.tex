\documentclass[a4paper,12pt]{article}
\usepackage{amsmath,amsfonts,amssymb}
\usepackage{graphicx}
\usepackage{float}

\begin{document}

\section*{Дальнейшее исследование основного неравенства}

Нами установлено, что число $t$ лежит в первой трети интервала $(p_n, q_n)$ при всех $n \geq 3$. Для того чтобы уточнить расположение числа $t$ в этом новом интервале, рассмотрим отношение $\sin \pi t / (\pi t)$ и $(p_n - q_n)$. Вычисления показывают (см. таблицы 1 и 2), что это отношение сильно, т.\,е. значения дробей:
$$ q_n - t = \frac{\pi}{p_n} - t \quad \text{и} \quad t - p_n = \frac{\pi}{q_n} - t, \quad n = 3, 6, 12, 24, $$
достаточно близки к 2. На основании этих вычислений мы с большой степенью уверенности можем считать, что в действительности имеет место соотношение:
$$ \lim_{n \to \infty} q_n - t = \frac{\pi}{2}. \tag{5} $$

Для доказательства соотношения (5) заметим, что (рис. 9):
$$ \sin \pi t = \sum_{k=1}^\infty \frac{(-1)^{k+1}(\pi t)^{2k-1}}{(2k-1)!}, $$
и, следовательно,
$$ \frac{q_n - t}{\cos \pi t} \approx \frac{\sin \pi t}{n}, $$
что даёт:
$$ x - t = \sin x < x - \frac{x^3}{6} + \frac{x^5}{120}, \tag{6} $$
существенно улучшая известное соотношение.

\subsection*{Неравенства (6) и (7)}
Для анализа можно использовать неравенство:
$$ 1 - \frac{x^2}{2} < \cos x < 1 - \frac{x^2}{2} + \frac{x^4}{24}. \tag{7} $$

Докажите их самостоятельно.

\subsection*{Таблицы}

\begin{table}[H]
\centering
\caption{Таблица 1}
\begin{tabular}{|c|c|c|}
\hline
$n$ & $p_n$ & $q_n$ \\ \hline
3   & 2.59807621 & 1.96152422 \\ \hline
6   & 3.46410162 & 3.19615242 \\ \hline
12  & 6.92820323 & 6.39230484 \\ \hline
24  & 13.85640646 & 12.78460968 \\ \hline
\end{tabular}
\end{table}

\begin{table}[H]
\centering
\caption{Таблица 2}
\begin{tabular}{|c|c|}
\hline
$n$ & $q_n - t - (p_n)$ \\ \hline
3   & 2.77773043 \\ \hline
6   & 2.27777730 \\ \hline
12  & 2.34543112 \\ \hline
\end{tabular}
\end{table}

\section*{Формула Гюйгенса и её эффективность}

Архимед использовал для вычисления числа $\pi$ приближенную формулу:
$$ \pi = \frac{3}{2} + \frac{1}{2} n^2 - \frac{1}{n^2}. $$

\end{document}